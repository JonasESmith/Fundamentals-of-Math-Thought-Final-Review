\documentclass[11pt]{article}
\usepackage{mathrsfs}
\usepackage{amsmath}
\usepackage{amssymb}
\usepackage{amsthm}
\usepackage[utf8]{inputenc}
\usepackage[english]{babel}

\newcommand{\numpy}{{\tt numpy}}    % tt font for numpy

\topmargin -.5in
\textheight 9in
\oddsidemargin -.25in
\evensidemargin -.25in
\textwidth 7in

\begin{document}

% ========== Edit your name here
\author{by Jonas Smith}
\title{Fundamentals of Math Thought Final Review}
\maketitle

\medskip

% ========== Begin answering questions here
\section*{Definitions}
\begin{enumerate}
    \item For integers a and b, define a $\mathbf{divides}$ b.
    \item Define a $\mathbf{proposition}$.
    \item Define a $\mathbf{conditional}$ $\mathbf{sentence}$.
    \item The $\mathbf{power}$ $\mathbf{set}$ of a set A.
    \item The $\mathbf{intersection}$ of sets A and B.
    \item The $\mathbf{union}$ of sets A and B.
    \item State one of $\mathbf{Demorgan's}$ $\mathbf{Laws}$ for two sets.
    \item An indexed family of sets $\mathscr{A}$ is $\mathbf{pairwise}$ $\mathbf{disjoint}$.
    \item State the Principle of Mathematical Induction.
    \item State the Well-Ordering Principle.
    \item A $\mathbf{relation}$ from A to B, for sets A and B.
    \item The $\mathbf{domain}$ of a relation R from A to B.
    \item The $\mathbf{range}$ of a relation R from A to B.
    \item The composite $S \circ R$ where R is a relation from A to B and S is a relation from B to C. 
    \item A relation R on A is $\mathbf{reflexive}$.
    \item A relation R on A is $\mathbf{symmetric}$.
    \item A relation R on A is $\mathbf{transitive}$.
    \item A $\mathbf{partition}$ $\mathscr{P}$ of a nonempty set A.
    \item A $\mathbf{function}$ from A to B, for sets A and B.
    \item $f:A\rightarrow B$ is $\mathbf{surjective}$
    \item $f:A\rightarrow B$ is $\mathbf{injective}$
    \item $f:A\rightarrow B$ and $Y \subseteq B$. Define $f^{-1}(Y).$
\end{enumerate}
\pagebreak

\section*{True False}
\begin{enumerate}
    \item 
\end{enumerate}
\pagebreak

\section{Part 3 Problems}
\begin{enumerate}
    \item 
\end{enumerate}
\pagebreak

\section{Part 4 Problems}
\begin{enumerate}
    \item 
\end{enumerate}
\pagebreak

\section{Proofs}
\begin{enumerate}
    \item Prove that $\sqrt{2}$ is irrational.
        \begin{proof}
            this is the beginning of a proof
        \end{proof}
        
    \item Prove that there are infinitely many prime numbers.
        \begin{proof}
        \end{proof}
        
    \item Let $x$ be an integer. Prove that if $x^2$ is not divisible by 4, then $x$ is odd.
        \begin{proof}
        \end{proof}
        
    \item When any integer $n$ is divided by 3, it has a remainder of 0, 1, or 2 ( by the division algorithm). This means that for any integer $n$ 
            \newline \newline
             $n = 3k for some integer k, or$
            \newline
            $n = 3k+1$ for some integer k, or 
            \newline
            $n = 3k+2$ for some integer.
    use this fact to prove that for any integer n, 3 divides $n^3 - n$.
        \begin{proof}
        \end{proof}
        
    \item For integers $a,b,c,x$ and $y$, prove that if $c$ divides $a$ and $c$ divides $b$, then $c$ divides $ax+by$.
        \begin{proof}
        \end{proof}
        
    \item For real numbers $x$ and $y$, prove that if $x$ is rational and $y$ is irrational, then $x + y$ is irrational. 
        \begin{proof}
        \end{proof}
        
    \item Let A, B, and C be sets. Prove that if $A \subseteq B,$ $B \subseteq C$, and $C \subseteq A$, then $A = B$ and $B = C$
        \begin{proof}
        \end{proof}
        
    \item Let A and B be sets. Prove that A = B if and only if $\mathscr{P}(A) = \mathscr{P}(B)$
        \begin{proof}
        \end{proof}
        
    \item Let A and B be sets. Prove that ($A^c \cup B)^c = A \cup B^c$
        \begin{proof}
        \end{proof}
        
    \item Let $A,B,C$, and $D$ be sets. Prove that if $C \subseteq A$ and $D \subseteq B,$ then $D-A\subseteq B-C.$
        \begin{proof}
        \end{proof}
        
    \item Use the Principle of Mathematical Induction to prove the following: \newline
    For all $n \in \mathbb{N}$
    \begin{center}
        $1\cdot1+2\cdot2!+3\cdot3!+\cdot \cdot \cdot + n\cdot n!=(n+1)!-1$
    \end{center}
        \begin{proof}
        \end{proof}
        
    \item Use the principle of Mathematical Induction to prove the following: For all $n \in \mathbb{N}:$
    \begin{center}
        8 divides $5^{2n} -1$
    \end{center}
        \begin{proof}
        \end{proof}
        
    \item We have the common differentiation formulas $\frac{d}{dx} x=1$ and $\frac{d}{dx}(fg)=f(\frac{d}{dx}g) + (\frac{d}{dx}f)g$. 
    \newline Use these formulas and the Principle of Mathematical Induction to prove that $\frac{d}{dx}x^n = nx^{n-1}$ for all $n \in \mathbb{N}$.
        \begin{proof}
        \end{proof}
        
    \item Let $a_1=2$, $a_2=4$, and $a_{n+2} = 5a_{n+1} - 6a_n$, for all $n \leq 1.$ Prove that $a_n = 2^n$ for all $n \in \mathbb{N}$
        \begin{proof}
        \end{proof}
        
    \item Let R be a relation from A to B and S be a relation from B to C. \newline
    Prove that $Rng(S\circ R) \subseteq Rng(S).$
        \begin{proof}
        \end{proof}
        
    \item Prove that if R is a symmetric, transitive relation on A and the domain of R is A, then R is reflexive on A. 
        \begin{proof}
        \end{proof}
        
    \item Let A be a nonempty set, $\mathscr{P}$ a partition of A, and B be a nonempty subset of A. \newline Prove that $\mathscr{A}=\{X \cap B$ $|$ $X \in \mathscr{P}$ $and$ $X \cap B \neq \varnothing \}$ is a partition of B.
        \begin{proof}
        \end{proof}
        
    \item Let R be an antisymmetric relation on the set A. \newnline Prove that if R is symmetric and $Dom(R) = A,$ then $R=I_A$
        \begin{proof}
        \end{proof}
        
    \item Define a relation R on $\mathbb{R}$ by $aRb$ \iff $a^3=b^5$. Prove that R is a function from $\mathbb{R}$ to $\mathbb{R}.$
        \begin{proof}
        \end{proof}
        
    \item Prove that, if $f$ and $g$ are functions, then $f \cap g$ is a function by showing that $f\capg = g|_{A}$ where \newline $A = \{x$ : $g(x) = f(x) \}.$
        \begin{proof}
        \end{proof}
        
    \item
        \begin{proof}
        \end{proof}
        
    \item
        \begin{proof}
        \end{proof}
        
    \item
        \begin{proof}
        \end{proof}
        
    \item
        \begin{proof}
        \end{proof}
        
    \item
        \begin{proof}
        \end{proof}
    
\end{enumerate}

\end{document}
\grid
\grid