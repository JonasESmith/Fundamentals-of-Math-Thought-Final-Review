\documentclass[11pt]{article}
\usepackage{mathrsfs}
\usepackage{amsmath}
\usepackage{amssymb}
\usepackage{amsthm}
\usepackage{enumerate}
\usepackage[utf8]{inputenc}
\usepackage[english]{babel}
\usepackage{indentfirst}
\renewcommand{\qedsymbol}{$\blacksquare$}

\newcommand{\numpy}{{\tt numpy}}    % tt font for numpy

\topmargin -.5in
\textheight 9in
\oddsidemargin -.25in
\evensidemargin -.25in
\textwidth 7in

\begin{document}

% ========== Edit your name here
\author{by Jonas Smith}
\title{Fundamentals of Math Thought Final Review}
\maketitle

\medskip

% ========== Begin answering questions here
\section*{Definitions}
\begin{enumerate}
    \item For integers a and b, define a $\mathbf{divides}$ b.
        \newline $\circ$ $Definition - $ there exists a $b$ such that $b=a\cdot k$, where $k \in \mathbb{Z}$
    \item Define a $\mathbf{proposition}$.
        \newline $\circ$ $Definition - $ A $\mathbf{proposition}$ is a sentence that has exactly one truth value.
    \item Define a $\mathbf{conditional}$ $\mathbf{sentence}$.
        \newline $\circ$ $Definition - $ For propositions P and Q, the $\mathbf{Conditional}$ $\mathbf{sentence}$ $P\implies Q$ is the proposition "If P, then Q". Proposition P is called the $\mathbf{antecednet}$ and Q is the $\mathbf{consequent}$.
    \item The $\mathbf{power}$ $\mathbf{set}$ of a set A.
        \newline $\circ$ $Definition - $ Let A be a set. The $\mathbf{power}$ $\mathbf{set}$ of A is the set whoe elements are the subsets of A and is denoted by $\mathscr{P}(A)$. Thus
        \begin{center}
            $\mathscr{P}(A) = \{B:B\subseteq A\}$
        \end{center}
    \item The $\mathbf{intersection}$ of sets A and B.
        \newline $\circ$ $Definition - $ The \textbf{intersection of A and B} is the set $A \cap B = \{ x:x \in A$ and $x \in B \}$.
    \item The $\mathbf{union}$ of sets A and B.
        \newline $\circ$ $Definition - $ The \textbf{union of A and B} is the set $A\cup B = \{ x:x \in A$ or $x\in B \}.$
    \item State one of $\mathbf{Demorgan's}$ $\mathbf{Laws}$ for two sets.
        \newline $\circ$ $Definition - $ $(A \cup B)^c = A^c \cap B^c.$ or $(A \cap B)^c = A^c \cup B^c.$
    \item An indexed family of sets $\mathscr{A}$ is $\mathbf{pairwise}$ $\mathbf{disjoint}$.
        \newline $\circ$ $Definition - $ The indexed family $\mathscr{A} = \{A_{\alpha}: \alpha \in \delta \}$ of sets is \textbf{pairwise disjoint} iff for all $\alpha$ and $\beta$ in $\delta$, either $A_\alpha = A_\beta$ or $A_\alpha \cap A_\beta = \varnothing.$
    \item State the Principle of Mathematical Induction.
    \begin{enumerate}[(i)]
        \item $1\in S,$ 
        \item for all $n\in \mathbb{N}$, if $n \in S$ then $n + 1 \in S.$
    \end{enumerate}
    Then S = $\mathbb{N}$
        
    \item State the Well-Ordering Principle.
        \newline $\circ$ $Definition - $ Every nonempty subset of $\mathbb{N}$ has a smallest element.
    \item A $\mathbf{relation}$ from A to B, for sets A and B.
        \newline $\circ$ $Definition - $ R is a relation from $a$ to $b$ iff R is a subset of $A$ X $B$
    \item The $\mathbf{domain}$ of a relation R from A to B.
        \newline $\circ$ $Definition - $ Dom(R)=$\{x\in A:$ there exists $y \in B$ such that $xRy \}$
    \item The $\mathbf{range}$ of a relation R from A to B.
        \newline $\circ$ $Definition - $ Rng(R)=$\{y\in A:$ there exists $x \in B$ such that $xRy \}$
    \item The composite $S \circ R$ where R is a relation from A to B and S is a relation from B to C. 
        \newline $\circ$ $Definition - $ $S \circ R = \{(a,c) :$ there exists $b \in B$ such that $(a,b) \in R$ and $(b,c) \in S \}$
    \item A relation R on A is $\mathbf{reflexive}$.
        \newline $\circ$ $Definition - $ iff for all $x \in A,$ $xRx$
    \item A relation R on A is $\mathbf{symmetric}$.
        \newline $\circ$ $Definition - $ iff for all $x,y \in A,$ $xRy$ then $yRx$
    \item A relation R on A is $\mathbf{transitive}$.
        \newline $\circ$ $Definition - $ iff for all $x,y,z \in A$, $xRy$ and $yRz$, then $xRz$
    \item A $\mathbf{partition}$ $\mathscr{P}$ of a nonempty set A.
        \newline $\circ$ $Definition - $ $\mathscr{P}$ is a partition of a nonempty set A.
        \begin{enumerate}[(i)]
            \item if $x\in \mathscr{P}$, then $x \neq \varnothing$
            \item if $x\in \mathscr{P}$ and $y \in \mathscr{P}$, then $x=y$ or $x \cap y = \varnothing$
            \item $\bigcup_{x = \mathscr{P}} x = A$
        \end{enumerate}
    \item A $\mathbf{function}$ from A to B, for sets A and B.
        \newline $\circ$ $Definition - $ is a relation $f:A\rightarrow B$ such that the $Dom(f) = A$, and if $(x,y) \in f$ and $(x,z) \in f$, then $y=z$
    \item $f:A\rightarrow B$ is $\mathbf{surjective}$
        \newline $\circ$ $Definition - $ A function $f:A\rightarrow B$ is onto B or $\mathbf{surjective}$ iff $Rng(f) = B$
    \item $f:A\rightarrow B$ is $\mathbf{injective}$
        \newline $\circ$ $Definition - $ A function $f:A \rightarrow B$ is one-to-one or is injective iff when $f(x)=f(y)$, then $x=y$.
    \item $f:A\rightarrow B$ and $Y \subseteq B$. Define $f^{-1}(Y).$
        \newline $\circ$ $Definition - $ $f^{-1}(Y) = \{x \in A$ : $y=f(x) \in Y\}$
\end{enumerate}
\pagebreak

\section*{True False}
\begin{enumerate}
    \item 
\end{enumerate}
\pagebreak

\section{Part 3 Problems}
\begin{enumerate}
    \item 
\end{enumerate}
\pagebreak

\section{Part 4 Problems}
\begin{enumerate}
    \item 
\end{enumerate}
\pagebreak

\section{Proofs}
\begin{enumerate}
    \item Prove that $\sqrt{2}$ is irrational.
        \begin{proof}[Proof:\nopunct] \\ 
             Assume $\sqrt{2}$ is rational. 
             \\ Thus $\sqrt{2} = \frac{a}{b}$, where $a,b \in \mathbb{Z}$ and $a,b$ have no common factors.
             \\ Now let us write $2 = \frac{a^2}{b^2}$, or $b^2=2a^2$.
             \\ Thus $b^2$ is even, the only way this can be true is that b itself is even.
             \\ hence $b^2$ is divisible by 4. This contradicts our assumption that a,b have no common factors
             \\ Therefore $\sqrt{2}$ cannot be rationalized.
        \end{proof}
        
    \item Prove that there are infinitely many prime numbers.
        \begin{proof}[Proof:\nopunct]
            Assume there exists a finite number of primes $\{p_1, p_2, ... , p_n \}$ 
            \\ Let $N = p_1\cdot p_2 \cdot p_3 \cdot \cdot \cdot p_{n+1}$
            \\ $N$ is not divisible by any of the known primes since it will leave a remainder of one upon division by any one of them.
            \\ Thus, N must be a divisible by some other prime not in our list
            \\ which contradicts the assumption that there is a finite number of primes.
            \\ Therefore There are an infinite number of prime numbers
        \end{proof}
        
    \item Let $x$ be an integer. Prove that if $x^2$ is not divisible by 4, then $x$ is odd.
        \begin{proof}[Proof:\nopunct] by contrapositive
            \begin{align*}
                & \text{ Assume x is not even.} \\
                & \text{ Then $x = 2k + 1$ for some $k\in \mathbb{Z}$ }\\
                & \text{ And if $x = 2k + 1$, it follows that }\\
                & \text{ $x^2 = (2k+1)^2= 4k^2 + 4k + 1 = 4(k^2+k)+1$}\\
                & \text{ Clearly, 4 does not divide $x^2=4(k^2+k)+1$, because 4 does not divide 1.}\\
                & \text{ Thus if x is not even $\implies$ $x^2$ is not divisible by 4.}\\
                & \text{ Clearly, if x is even $\implies$ $x^2$ is divisible by 4.}
            \end{align*}
        \end{proof}
        
    \item When any integer $n$ is divided by 3, it has a remainder of 0, 1, or 2 ( by the division algorithm). This means that for any integer $n$ 
            \newline \newline
             $n = 3k$ for some integer k, or
            \newline
            $n = 3k+1$ for some integer k, or 
            \newline
            $n = 3k+2$ for some integer.
    use this fact to prove that for any integer n, 3 divides $n^3 - n$.
        \begin{proof}[Proof:\nopunct]
         Let $n=3k$ where $k\in \mathbb{Z}$. Then
           \begin{align*}
               n^3-n & = (3k)^3-3k \\ 
                     & =27k^3-3k \\
                     & = 3(9k^3 - k), 9k^3-k \in \mathbb{Z} \text{, and is divisible by 3}
           \end{align*}
           
           \pagebreak
           Now let $n=3k+1$
           \begin{align*}
               \text{Then, } n^3-n & = (3k+1)^3-(3k+1) \\ 
                     & = (9k^2+6k+1)(3k+1)-(3k+1) \\
                     & = 27k^3 + 18k^2+2k+9k^2+6k+1-9k-1 \\
                     & = 27k^3 + 27k^2+6k \\
                     & = 3(9k^3+9k^2+2k) \\
                     & = (9k^3+9k^2+2k) \in \mathbb{Z} \text{, and} \\
                     & = 3(9k^3+9k^2+2k) \text{ is divisible by 3.}
           \end{align*}
           
           Now let $n=3k+2$
           \begin{align*}
               \text{Then, } n^3-n & = (3k+2)^3+(3k+2) \\
                                & = (9k^2+12k+4)(3k+2) - (3k+2) \\
                                & = (27k^3+36k^2+12k+18k^2+24k+9-3k-2 \\
                                & = 27k^3+54k^2+33k+6 \\
                                & = 3(9k^3+18k^2+11k+2) \\
                                & = (9k^3+18k^2+11k+2) \in \mathbb{Z} \text{, and} \\
                                & = 3(9k^3+18k^2+11k+2) \text{ is divisible by 3.}
           \end{align*}
           
           Therefore for any integer $n = 3k,$ $ n = 3k+1, \text{ or }n=3k+2$, $n$ is divisible by 3.
            
        \end{proof}
        
    \item For integers $a,b,c,x$ and $y$, prove that if $c$ divides $a$ and $c$ divides $b$, then $c$ divides $ax+by$.
        \begin{proof}[Proof:\nopunct]
        \begin{align*}
            & \text{Let $a,b,c,x,  \text{ and } y, \in \mathbb{Z}$,} \\
            & \text{Let c divide b such that $b=c \cdot k$, where $k \in \mathbb{Z}$, and} \\
            & \text{Let c divide a such that $a=c \cdot m$, where $m \inf \mathbb{Z}$} \\
            & \text{now let $\frac{ax+by}{c}=\frac{(ck)x+(cm)y}{c}=\frac{c}{c}(kx+my)=(kx+my)$} \\
            & \text{Therefore c divides $ax+by$}
        \end{align*}
        \end{proof}
        
    \item For real numbers $x$ and $y$, prove that if $x$ is rational and $y$ is irrational, then $x + y$ is irrational. 
        \begin{proof}[Proof:\nopunct]
            \begin{align*}
                &\text{Let x be rational, and y be irrational} \\
                & \text{Let $x+y$} \text{ be rational, then $x+y=\frac{a}{b}$. Let $x=\frac{c}{d}$. Then} \\
                & \text{y = $\frac{a}{b} - x=\frac{a}{b}-\frac{c}{d}=\frac{ad-bc}{bd}$} \\
                & \text{Thus y is rational which contradicts our first assumption that y is irrational.} \\ & \text{Hence it follows that x + y is irrational}
            \end{align*}
        \end{proof}
        
    \item Let A, B, and C be sets. Prove that if $A \subseteq B,$ $B \subseteq C$, and $C \subseteq A$, then $A = B$ and $B = C$
       \begin{proof}[Proof:\nopunct]
            \begin{align*}
                \text{Let A} & \text{ $\subseteq B, B \subseteq C$, and $C \subseteq A$,} \\
                & \text{and let $x\in  A$, then by the definition of $\subseteq$, $x\in B,$ and $x\in C$} \\
                & \text{Thus $A\subseteq B \subseteq C$.} \\
                & \text{and in order for $A \subseteq B$, and $B \subseteq C$, and $C\subseteq A$} \\
                & \text{Therefore $A=B=C.$} \\
            \end{align*}
        \end{proof}
        
    \item Let A and B be sets. Prove that A = B if and only if $\mathscr{P}(A) = \mathscr{P}(B)$
        \begin{proof}
            \begin{align*}
                & \text{Let A and B be sets} \\
                & \text{Suppose A = B} \\
                & \text{Since $\mathscr{P}(A)$ is the set of all subsets of A and $\mathscr{P}(B)$ is the set of all subsets of B} \\
            \end{align*}
        \end{proof}
        
    \item Let A and B be sets. Prove that ($A^c \cup B)^c = A \cup B^c$
        \begin{proof}[Proof:\nopunct]
        \end{proof}
        
    \item Let $A,B,C$, and $D$ be sets. Prove that if $C \subseteq A$ and $D \subseteq B,$ then $D-A\subseteq B-C.$
        \begin{proof}[Proof:\nopunct]
        \end{proof}
        
    \item Use the Principle of Mathematical Induction to prove the following: \newline
    For all $n \in \mathbb{N}$
    \begin{center}
        $1\cdot1+2\cdot2!+3\cdot3!+\cdot \cdot \cdot + n\cdot n!=(n+1)!-1$
    \end{center}
        \begin{proof}[Proof:\nopunct]
        \end{proof}
        
    \item Use the principle of Mathematical Induction to prove the following: For all $n \in \mathbb{N}:$
    \begin{center}
        8 divides $5^{2n} -1$
    \end{center}
        \begin{proof}[Proof:\nopunct]
        \end{proof}
        
    \item We have the common differentiation formulas $\frac{d}{dx} x=1$ and $\frac{d}{dx}(fg)=f(\frac{d}{dx}g) + (\frac{d}{dx}f)g$. 
    \newline Use these formulas and the Principle of Mathematical Induction to prove that $\frac{d}{dx}x^n = nx^{n-1}$ for all $n \in \mathbb{N}$.
        \begin{proof}[Proof:\nopunct]
        \end{proof}
        
    \item Let $a_1=2$, $a_2=4$, and $a_{n+2} = 5a_{n+1} - 6a_n$, for all $n \leq 1.$ Prove that $a_n = 2^n$ for all $n \in \mathbb{N}$
        \begin{proof}[Proof:\nopunct]
        \end{proof}
        
    \item Let R be a relation from A to B and S be a relation from B to C. \newline
    Prove that $Rng(S\circ R) \subseteq Rng(S).$
        \begin{proof}
        \end{proof}
        
    \item Prove that if R is a symmetric, transitive relation on A and the domain of R is A, then R is reflexive on A. 
     \begin{proof}[Proof:\nopunct]
        \end{proof}
        
    \item Let A be a nonempty set, $\mathscr{P}$ a partition of A, and B be a nonempty subset of A. \newline Prove that $\mathscr{A}=\{X \cap B$ $|$ $X \in \mathscr{P}$ $and$ $X \cap B \neq \varnothing \}$ is a partition of B.
        \begin{proof}[Proof:\nopunct]
        \end{proof}
        
    \item Let R be an antisymmetric relation on the set A. 
    \newnline Prove that if R is symmetric and $Dom(R) = A,$ then $R=I_A$
       \begin{proof}[Proof:\nopunct]
        \end{proof}
        
    \item Define a relation R on $\mathbb{R}$ by $aRb$ $\iff$ $a^3=b^5$. Prove that R is a function from $\mathbb{R}$ to $\mathbb{R}.$
       \begin{proof}[Proof:\nopunct]
        \end{proof}
        
    \item Prove that, if $f$ and $g$ are functions, then $f \cap g$ is a function by showing that $f\capg = g|_{A}$ where \newline $A = \{x$ : $g(x) = f(x) \}.$
       \begin{proof}[Proof:\nopunct]
        \end{proof}
        
    \item Prove that if $f:A \xrightarrow{\text{onto}} B$ and $g:B\xrightarrow{\text{onto}} C$, then $g\circ f:A \xrightarrow{\text{onto}} C.$
        \begin{proof}[Proof:\nopunct]
        \end{proof}
        
    \item Let $f:A\xrightarrow{} B$. Prove that if $f^{-1}$ is a function, then $f$ is injective.
       \begin{proof}[Proof:\nopunct]
        \end{proof}
        
    \item Prove that $g:(-\infty, -4)\rightarrow(-\infty,0)$, defined by $g(x) = -|x+4|$ is a one-to-one correspondence.
       \begin{proof}[Proof:\nopunct]
        \end{proof}
        
    \item Let $f:A \rightarrow B$ and $Y \subseteq B.$ Prove that $f(f^{-1}(Y)) \subseteq Y.$
        \begin{proof}[Proof:\nopunct]
        \end{proof}
        
    \item Let $f:A \rightarrow B$ be injective and $X \subseteq A.$ \newline Prove that $X = f^{-1}(f(X))$ 
        \begin{proof}[Proof:\nopunct]
        \end{proof}
    
\end{enumerate}

\end{document}
\grid
\grid