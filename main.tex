\documentclass[11pt]{article}
\usepackage{mathrsfs}
\usepackage{amsmath}
\usepackage{amssymb}
\usepackage{amsthm}
\usepackage[utf8]{inputenc}
\usepackage[english]{babel}

\newcommand{\numpy}{{\tt numpy}}    % tt font for numpy

\topmargin -.5in
\textheight 9in
\oddsidemargin -.25in
\evensidemargin -.25in
\textwidth 7in

\begin{document}

% ========== Edit your name here
\author{Jonas Smith}
\title{Fundamentals of Math Thought Final Review}
\maketitle

\medskip

% ========== Begin answering questions here
\section*{Definitions}
\begin{enumerate}
    \item For integers a and b, define a $\mathbf{divides}$ b.
    \item Define a $\mathbf{proposition}$.
    \item Define a $\mathbf{conditional}$ $\mathbf{sentence}$.
    \item The $\mathbf{power}$ $\mathbf{set}$ of a set A.
    \item The $\mathbf{intersection}$ of sets A and B.
    \item The $\mathbf{union}$ of sets A and B.
    \item State one of $\mathbf{Demorgan's}$ $\mathbf{Laws}$ for two sets.
    \item An indexed family of sets $\mathscr{A}$ is $\mathbf{pairwise}$ $\mathbf{disjoint}$.
    \item State the Principle of Mathematical Induction.
    \item State the Well-Ordering Principle.
    \item A $\mathbf{relation}$ from A to B, for sets A and B.
    \item The $\mathbf{domain}$ of a relation R from A to B.
    \item The $\mathbf{range}$ of a relation R from A to B.
    \item The composite $S \circ R$ where R is a relation from A to B and S is a relation from B to C. 
    \item A relation R on A is $\mathbf{reflexive}$.
    \item A relation R on A is $\mathbf{symmetric}$.
    \item A relation R on A is $\mathbf{transitive}$.
    \item A $\mathbf{partition}$ $\mathscr{P}$ of a nonempty set A.
    \item A $\mathbf{function}$ from A to B, for sets A and B.
    \item $f:A\rightarrow B$ is $\mathbf{surjective}$
    \item $f:A\rightarrow B$ is $\mathbf{injective}$
    \item $f:A\rightarrow B$ and $Y \subseteq B$. Define $f^{-1}(Y).$
\end{enumerate}
\pagebreak

\section*{True False}
\begin{enumerate}
    \item 
\end{enumerate}
\pagebreak

\section{Part 3 Problems}
\begin{enumerate}
    \item 
\end{enumerate}
\pagebreak

\section{Part 4 Problems}
\begin{enumerate}
    \item 
\end{enumerate}
\pagebreak

\section{Proofs}
\begin{enumerate}
    \item Prove that $\sqrt{2}$ is irrational.
        \begin{proof}
            this is the beginning of a proof
        \end{proof}
    \item Prove that there are infinitely many prime numbers.
        \begin{proof}
        \end{proof}
    \item Let $x$ be an integer. Prove that if $x^2$ is not divisible by 4, then $x$ is odd.
        \begin{proof}
        \end{proof}
    \item When any integer $n$ is divided by 3, it has a remainder of 0, 1, or 2 ( by the division algorithm). This means that for any integer $n$
        \begin{center}
            $n = 3k for some integer k, or$
        \end{center}
        \begin{center}
            $n = 3k+1$ for some integer k, or 
        \end{center}
        \begin{center}
            $n = 3k+2$ for some integer.
        \end{center}
    use this fact to prove that for any integer n, 3 divides $n^3 - n$.
        \begin{proof}
        \end{proof}
    \item For integers $a,b,c,x$ and $y$, prove that if $c$ divides $a$ and $c$ divides $b$, then $c$ divides $ax+by$.
        \begin{proof}
        \end{proof}
\end{enumerate}

\end{document}
\grid
\grid